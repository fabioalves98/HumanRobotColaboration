\chapter{Conclusion}





\section{Overview}
% Descrição inicial do que foi proposto
\par This Dissertation proposed a set of tools and techniques that when joined together, promoted the creation and execution of collaborative tasks between a robotic manipulator and a human. The initial techniques proposed consisted on interacting with the robot through touch, converting the \ac{ft} measured at the \ac{eef} into robot motion, compensate extra weight coupled to the \ac{eef}, and detect moving obstacles in the environment. With these techniques, a set of tasks was proposed that included precise \ac{hg} of the robot, transfer of tools between the robot and the human, manipulation of heavy objects, and collision free execution of an industrial task.

% Resumo do que foi feito no capitulo de compensação do sensor de FT
\par These tasks  heavily relied on the existence of \ac{ft} sensing capabilities on the robot, so we started by exploring the capabilities and behavior of the \ac{ft} sensor located on its \ac{eef}. Some unexpected behavior was found, such as the unreasonable variation of \ac{ft} caused by the position of the last joint. This behavior and other peculiarities were properly corrected. Then, to be able to have multiple tools and objects attached to the \ac{eef}, and still be able to precisely \ac{hg} it, a payload compensation model was developed based on the optimization of an analytical \ac{ft} generation model. The combination of the active correction of \ac{ft} measurements based on the Wrist3 Joint, and the compensation based on the measured payload, coupled to the robot, allowed the user to precisely \ac{hg} the robot with multiple tool configurations and objects.

% Resumo do que foi feito no capitulo de collision avoidance
\par To increase the collaborativeness and safety of the system, an external vision sensor was added to the workspace in order to give the robot obstacle avoidance capabilities. To achieve this, a motion controller based on the \ac{apf} method was developed. In this approach, an attraction vector was generated, that made the robot follow a predefined offline trajectory. Then, a repulsion vector was created from the real time identification of obstacles in the environment. The combination of these 2 components allowed the robot to adapt its motion to the existence of dynamic obstacles, and still be able to complete its trajectory.

% Resumo do que foi feito no capítulo de collaborative tasks
\par With this abilities, a group of collaborative tasks was developed and seamlessly combined in a state machine. The tasks included transfer of objects between the human and the robot, precisely \ac{hg} the robot at the \ac{eef} level, dynamically couple an object and manipulate it with the same motion as in the previous task, and the execution of a predefined industrial task with the ability of avoiding dynamic obstacles.

% Resumo do que foi feito no capitulo de de experiments and results
\par To test and validate the performance of the proposed tasks, a number of experiments were designed and implemented in a real scenario with a \ac{ur10e} cobot. The tests consisted on the execution of the various tasks with different system parameters. Some metrics were registered based on the ability to complete the tasks and the qualitative performance of the system, while doing so. The system showed correct behavior in every task and the results show that the best possible behavior relies on proper system parametrization.

% Dificuldades ao longo da dissertação
\par During the development of this Dissertation, there were some challenges not initially forseen when drawing its requirements. For starters, the amount of inaccuracies in the \ac{ft} sensor was not expected, and for some time, it was uncertain how much the collaborative tasks could rely on it. It was only until the right tests were made and the error patterns discovered, that a possible correction was designed and implemented. Another struggle came from the fact that these inaccuracies could only be tested with the real \ac{ft} sensor. Regarding the control of the \ac{ur10e}, the journey to find the best method for joint velocity based control of the robot was also long and uncertain. For instance, the \textit{iris\_ur10e} \ac{ros} package was used in the initial periods of development. It consisted on a fork of the official \ac{ur} \ac{ros} driver with some adicional features, such as the inclusion of the Weiss Gripper driver and description. But because this package forked an old version of the driver, joint velocity based control was not available, which raised the question of which control interface to use, given the options outlined in \autoref{ssec:ur10e}. Ultimately, efforts were made to merge the newest features of the official \ac{ur} \ac{ros} driver with the implementation characteristics of the setup at IRISLab.

% Final remark
\par Nevertheless, every proposed technique was effectively implemented, and every goal successfully achieved. The final collaboration framework allows the safe execution of predefined tasks while seamlessly interfacing with the robot in many different ways. This work can be used as a whole for diverse applications, but also contributes in specific topics on the field of \ac{hrc} with the development of the payload compensation model, the \ac{apf} collision avoidance controller, the collaborative state machine, and the generic \acp{gui} for robot monitoring and control. Finally, regarding these achievements and contributions, most importantly, this work helps breaking the barriers that are currently separating humans and industrial robots.





\section{Future Work}

% Introdução às várias formas de future work
\par Despite every proposed goal being accomplished, \ac{hrc} is a vast field and this Dissertation only covered a small percentage of its intricacies, therefore, space for added functionality and improved performance is plenty. In no particular order, a few key aspects in which this work can be expanded will be outlined:

\begin{itemize}
    \item \textbf{Full Body \ac{hg}: }The flexibility of manipulating an industrial robot with physical interaction can be enhanced if extended to every single joint. In this work, the \ac{hg} of the robot was limited to the \ac{ft} measured at the \ac{eef}. There are scenarios where being able to \ac{hg} the entire robot is very useful, such as when the movement of a single joint is desired, or when the robot is reaching a singularity.
    \item \textbf{Extended Sensor Apparatus: }Having a single vision sensor is only sufficient for a proof of concept collision avoidance model, but is not enough for a real industry scenario. Increasing the number of sensors, playing with its dispositions and testing multiple configuration seems a natural evolution on the work developed. An interesting ideia would be to implement a vision sensor in the structure of the robot.
    \item \textbf{Multimodal Interaction: }The proposed work provides limited ways of conveying actions to the robot. There are multiple methods that can be added to this system, in order to increase the means of communication with the robot such as voice recognition, gesture recognition, high level task interpretation, or simply increasing the number of inputs of the system with hardware buttons and switches.
    \item \textbf{High Level Creation of Tasks: }The \textit{ros\_smach} library allows the creation of complex robot behavior, describing it as structured state machines. To extend the existing state machine with new functionalities, knowledge in Python programming is required. It would be useful if it was possible to create new tasks without programming knowledge. An intuitive \ac{gui} could be built for this purpose, with a drag and drop interface and a list of robot skills that the user could choose from.
    \item \textbf{Offline Trajectory Refactoring: }In the implemented approach to collision avoidance, the offline trajectory is static throughout its execution. There are scenarios with an high amount of obstacles where the robot is not able to complete the trajectory. This behavior could be avoided by making local changes to the trajectory or, in extreme cases, by globally replanning the trajectory, but this time, including the identified obstacles in the planning environment.
\end{itemize}
