\chapter{Introduction}
\label{chp:1-intro}

\par In this chapter, we contextualize this Dissertation and present its topics. We start with a concise motivation that enlightens the broad necessity of \ac{hrc}. Then, we specify the goals we want to achieve, and what exactly has been created in this work. Finally, the thesis outline is given, with a short summary of each chapter contents.





\section{Motivation}

\par An industrial robot is a machine that allows efficiency, strength and automation on the production line. It can be seen as a substitute for the human worker for their ability to perform repetitive and tedious manufacturing tasks autonomously with high accuracy and precision. By default, and for safety reasons they are intended to work separately from humans, in their own environment and are usually programmed to stop if a human worker enters its space.

\par Despite their great abilities, there are some tasks that are either too complex, or too dynamic to fully automate and require the continuous presence of a human, to either supervise or assist the robot. This need of shared execution of tasks with robots, resulted in the emergence of Collaborative Robots, also known as \acsp{cobot}. These are industrial robots generally built with lightweight materials, equipped with \ac{ft} assistance and speed limitation hardware, and overall, are easy to set up and program. Altogether these features promote safe and efficient interactions with humans.

\par Starting with a cobot, to reach the collaborative execution of a task, there is a gap which this Dissertation aims to fill, by proposing and developing a set of tools, techniques and workflows that not only leverage the hardware of cobots, but also use external sensors. Such techniques are then used to create and deploy multiple tasks which, when grouped and integrated in an interaction based state machine, result in the promotion of a shared collaborative environment where humans and cobots can safely work together. 





\section{Objectives}

\par This Dissertation has as its main objective the promotion of shared execution of tasks, between humans and industrial robots, specifically targeting collaborative robotic manipulators. To do so, it aims to develop the following set of techniques:

% TODO Recheck techniques and tasks 

\begin{itemize}
    \item Interaction and communication with cobot through touch;
    \item Manipulation of the cobot by physically \ac{hg} it;
    \item Dynamic compensation of weight coupled to the \ac{eef};
    \item Detection of dynamic obstacles in the environment;
    \item Safe motion planning aware of dynamic obstacles;
    \item Creation of high-level task plans;
\end{itemize}

\par With theses techniques, this Dissertation also aims to achieve the following collaborative tasks:

\begin{itemize}
    \item Precise \ac{hg} at the \ac{eef} level;
    \item Transfer of tools and objects between human and cobot;
    \item Lift assistance and precise manipulation of heavy objects;
    \item Collision free execution of an industrial task;
\end{itemize}

\par Besides the main objectives, efforts were made to increase the ease of use of a robotic manipulator, with the development of 2 graphical toolbox interfaces, which help control and monitor the status of the cobot and the task being executed. Finally, all techniques were deployed in a modular task-level architecture, which allows the user to rapidly implement new collaborative tasks and easily monitor their execution.





\section{Outline}

\par This Dissertation is divided into 7 chapters.
\par \autoref{chp:2-collab-robots} (Collaborative Robotics) enlightens the theoretical background that supports this work. It starts with the historical context and various definitions on the \ac{hrc} field. Some technologies used for the development of cobotic applications are outlined and specific details are given on the hardware used in this work. Then, it gives an analysis on existing research proposals done in this field.

\par \autoref{chp:3-ft-sensor-correction} (Force Torque Sensor Compensation) explains how a 6-Axis \ac{ft} sensor located at the cobot \ac{eef} can be leveraged for precise \ac{hg} applications. It also presents a payload compensation system based on a theoretical \ac{ft} model, that is used to separate forces caused by the human from forces caused by attached payload. Finally, it describes a real time architecture for correction and compensation of \ac{ft}.

\par \autoref{chp:4-obstacle} (Dynamic Obstacle Avoidance) demonstrates how a depth camera can be integrated with a cobot and make its motion planning collision aware. It outlines techniques for point cloud segmentation that correctly identify moving obstacles in the environment. Then, it presents a collision avoidance system that takes into account the identified obstacles when executing predefined trajectories. Just like the previous chapter, it also described a real time architecture that implements these models.

\par \autoref{chp:5-tasks} (Collaborative Tasks) uses the previous architectures to build the proposed collaborative tasks. Then, it aggregates all tasks in a collaborative state machine where a human can seamlessly execute and switch between tasks through touch interaction. It also showcases 2 \ac{gui} designed for the development of robotic applications.

\par \autoref{chp:6-tests} (Experiments and Results) demonstrates the performance of the tasks created though various metrics obtained from experiments made, both in simulated and real environments.

\par \autoref{chp:7-conclusion} (Conclusion) gives the final regards about the work developed and elaborates on future work.
