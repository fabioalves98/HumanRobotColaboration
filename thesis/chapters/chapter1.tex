\chapter{Introduction}
\label{chp:1-intro}

% TODO Add intro to chapter 1

\section{Motivation}

\par An industrial robot is a machine that allows efficiency, strength and automation on the production line. It can be seen as a substitute for the human worker for their ability to perform repetitive and tedious manufacturing tasks autonomously with high accuracy and precision. By default, and for safety reasons they are intended to work separately from humans, in their own environment and are usually programmed to stop if a human worker enters its space.

\par Despite their great abilities, there are some tasks that are either too complex, or too dynamic to fully automate and require the continuous presence of a human, to either supervise or assist the robot. This need of shared execution of tasks with robots, resulted in the emergence of Collaborative Robots, also known as \acsp{cobot}. These are industrial robots generally built with lightweight materials, equipped with \ac{ft} assistance and speed limitation hardware, and overall, are easy to set up and program. Altogether these features promote safe and efficient interactions with humans.

\par Starting with a cobot, to reach the collaborative execution of a task, there is a gap which this Dissertation aims to fill, by proposing a set of tools, techniques and workflows that not only leverage on the hardware of cobots, but also use external sensors to promote a shared collaborative environment where humans and cobots can safely execute tasks together.

% TODO Complete motivation. Say the UR10e will be used

\section{Objectives}

\par This Dissertation has as its main objective the promotion of shared execution of tasks, between humans and industrial robots, specifically targeting collaborative robotic manipulators. To do so, it aims to develop the following set of techniques:

% TODO Rename techniques and tasks 

\begin{itemize}
    \item Interaction and communication with cobot through touch;
    \item Manipulation of the cobot by physically \ac{hg} it;
    \item Dynamic compensation of weight coupled to the \ac{eef};
    \item Detection of dynamic obstacles in the environment;
    \item Safe motion planning aware of dynamic obstacles;
    \item Creation of high-level task plans;
\end{itemize}

\par With theses techniques, this Dissertation also aims to achieve the following collaborative tasks:

\begin{itemize}
    \item Precise \ac{hg} at the \ac{eef} level;
    \item Transfer of tools and objects between human and cobot;
    \item Lift assistance and precise manipulation of heavy objects;
    \item Collision free execution of an industrial task;
\end{itemize}

\par Besides the main objectives, efforts were made to increase the ease of use of a robotic manipulator, with the development of 2 graphical toolbox interfaces, which help control and monitor the status of the cobot and the task being executed. Finally, all techniques were deployed in a modular task-level architecture, which allows the user to rapidly implement new collaborative tasks and easily monitor their execution.

\section{Outline}

% TODO Restructure the chapter descriptions 

\par This Dissertation is divided into 7 chapters.
\par Chapter 2 (Collaborative Robotics) starts by giving an introductory context on the field of Human-Robot Collaboration, its relevance, achievements and problems. Then, it gives an analysis on existing approaches and technologies, both commercial and open-source. It concludes by discussing the framing of this Dissertation inside this field, what it aims to solve and contribute.
\par Chapter 3 (Force Torque Sensor Compensation) explains the techniques developed that are cobot-centered. It shows how the forces and torques felt at the \ac{eef} of the cobot are used to achieve \ac{hg} with weight compensation.
\par Chapter 4 (Dynamic Obstacle Avoidance) explains the techniques developed that are environment-centered, such as the use of an external depth camera to detect obstacles, and the implementation of a online collision avoidance algorithm.
\par Chapter 5 (Collaborative Tasks) showcases the use of the previously developed techniques to execute a set of tasks between a human and a cobot.
\par Chapter 6 (Experiments and Results) presents the experiments made to validate the techniques and the respective results.
\par Chapter 7 (Conclusion) gives the final regards about the work developed and elaborates on future work.
